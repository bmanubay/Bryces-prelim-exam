\RequirePackage{fix-cm}
\RequirePackage[hyphens]{url}
\RequirePackage[final]{graphicx} % need to show figures in draft mode
\documentclass[aps,pre,onecolumn,nofootinbib,superscriptaddress,linenumbers,12pt, draft,tightenlines]{revtex4-1}


% Change to a sans serif font.
\usepackage{sourcesanspro}
\renewcommand*\familydefault{\sfdefault} %% Only if the base font of the document is to be sans serif
\usepackage[T1]{fontenc}
%\usepackage[font=sf,justification=justified]{caption}
\usepackage[font=sf]{floatrow}

% Rework captions to use sans serif font.
\makeatletter
\renewcommand\@make@capt@title[2]{%
 \@ifx@empty\float@link{\@firstofone}{\expandafter\href\expandafter{\float@link}}%
  {\textbf{#1}}\sf\@caption@fignum@sep#2\quad
}%
\makeatother

%\linespread{0.956}

\usepackage{listings} % For code examples
\usepackage[usenames,dvipsnames,svgnames,table]{xcolor}
\usepackage{amsmath}
\usepackage{amssymb}
\usepackage{graphicx}
\usepackage{dcolumn}
\usepackage{boxedminipage}
\usepackage[colorlinks=true,citecolor=blue,linkcolor=blue]{hyperref}
\usepackage[]{microtype}
\usepackage[obeyFinal]{todonotes}
\usepackage{import}
\usepackage{setspace, siunitx, amsmath,amsfonts, adjustbox,booktabs, cleveref}
%\usepackage{caption}
\usepackage{subcaption}
\usepackage{enumitem}
\usepackage{titlesec}
\usepackage{enumitem}
\setcounter{secnumdepth}{5}

% Units
\DeclareSIUnit\Molar{\textsc{m}}


% Comments
\newcounter{comment}
\newcommand{\comment}[2][]{%
% initials of the author (optional) + note in the margin
\refstepcounter{comment}%
{%
\setstretch{0.7}% spacing
\todo[inline, color={cyan!45},size=\small]{%
\textbf{\footnotesize [\uppercase{#1}\thecomment]:}~#2}%
}}

% Start supplementary sections

\newcommand{\beginsupplement}{%
        \onecolumngrid
        \setcounter{table}{0}
        \renewcommand{\thetable}{S\arabic{table}}%
        \setcounter{figure}{0}
        \renewcommand{\thefigure}{S\arabic{figure}}%
     }

\graphicspath{{figures/}}
\floatsetup[table]{capposition=top}
\begin{document}

%\documentclass[a4paper,12pt]{article}
%\usepackage[superscript,biblabel]{cite}
%\begin{document}

%%%%%%%%%%%%%%%%%%%%%%%%%%%%%%%%%%%%%%%%%%%%%%%%%%%%%%%%%%%%%%%%%%%%%%%%%%%%%%%%
% DOCUMENT
%%%%%%%%%%%%%%%%%%%%%%%%%%%%%%%%%%%%%%%%%%%%%%%%%%%%%%%%%%%%%%%%%%%%%%%%%%%%%%%%

\title{A method for redesigning molecular mechanics force field parameterization by use of a Bayesian statistical framework}

\author{Bryce C. Manubay} 
\email{bryce.manubay@colorado.edu}
\affiliation{University of Colorado - Department of Chemical and Biological Engineering}

% Date
\date{\today}

%%%%%%%%%%%%%%%%%%%%%%%%%%%%%%%%%%%%%%%%%%%%%%%%%%%%%%%%%%%%%%%%%%%%%%%%%%%%%%%%
% ABSTRACT
%%%%%%%%%%%%%%%%%%%%%%%%%%%%%%%%%%%%%%%%%%%%%%%%%%%%%%%%%%%%%%%%%%%%%%%%%%%%%%%%

\begin{abstract}
This document describes a collected set of best practices for computing various physical properties from molecular simulations of liquid mixtures.

\emph{Keywords: best practices; molecular dynamics simulation; physical property computation}


\end{abstract}
\maketitle

\listoftodos

\section{Outline}
\subsection{Objectives $\left(0.5 pages\right)$}
\begin{itemize}
 \item Molecular dynamics simulation is becoming more integral to many scientific studies
  \begin{itemize}
   \item Observing physical phenomena at a molecular scale (phase changes, ligand docking, etc.)\cite{villin}
   \item Drug discovery and deisgn of new molecules\cite{drug_discov}
  \end{itemize}
 \item The necessity for transferable and accurate force fields is therefore imperative
  \begin{itemize}
   \item Transferability encourages use and provides convenience for scientists with wide arrays of research interests
   \item Inaccurate force fields have been shown to grossly misrepresent even somewhat simple molecular systems\cite{ffcomp1,ffcomp2}
  \end{itemize}
 \item A possible solution to the given problem can be by recasting the force field parameterization issue as a bayesian inference problem
 \item The objective of this paper is introduce a framework for using disparate experimental data in order to parameterize molecular mechanics force fields
 \item In this paper I will describe the process of collecting and organizing large amounts of high quality thermochemical data and preliminary use of the Multistate Bennett Acceptance Ratio (MBAR) as a means to improve sampling speed during the parameterization process.      
\end{itemize}

\subsection{Significance $\left(0.5 pages\right)$}

\subsection{Background and related literature $\left(1.5 pages \pm 0.5 pages\right)$}

\subsection{Methods $\left(1.5 pages \pm 0.5 pages\right)$}

\subsection{Progress $\left(1.5 pages \pm 0.5 pages\right)$}

\subsection{Research plan $\left(0.5 pages\right)$}

\bibliographystyle{prsty}
\bibliography{report_outline}
\end{document}
